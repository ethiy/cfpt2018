\documentclass[a4paper,french]{article}
\usepackage{rfiap2018_resume}

\usepackage[utf8]{inputenc}
\usepackage[T1]{fontenc}
\usepackage{babel}
\usepackage{times}

\begin{document}
    \date{}
    \title{
        \Large\bf Autoqualification de modèle 3D de bâtiments pour l'évaluation de méthode de reconstruction.
    }
    \author{
        \begin{tabular}[t]{c@{\extracolsep{4em}}c@{\extracolsep{4em}}c@{\extracolsep{4em}}c}
            Oussama Ennafii${}^1$ & Clément Mallet${}^1$ & Arnaud Le-Bris${}^1$ & Florent Lafarge${}^2$ \\
        \end{tabular}
        {}\\
        \\
        ${}^1$        Univ. Paris-Est, LaSTIG MATIS, IGN, ENSG, F-94165 Saint-Mandé, France\\
        ${}^2$        Inria, Titane, 06902 Sophia Antipolis, France
        {}\\
        \\
        oussama.ennafii@ign.fr\\
    }
    \maketitle
    \thispagestyle{empty}

    \begin{abstract}
        La reconstruction automatique de scènes urbaines est l'objet d'intérêt de la communauté scientifique autant que les industriels. Cependant, même les derniers algorithmes de modélisation urbaines ne sont pas viables du point de vue opérationel. Nous proposons donc une méthode de qualification automatique de modèle de bâtiments 3D. Une taxonomy, qui hierarchise les erreurs possibles que l'on peut relever sur un modèle 3D urbain, est proposée. Le problème de qualification devient ainsi un problème de classification supervisée. Des attributs sont calculées à partir de la forme géométrique du modèle, aussi qu'en se basant sur des Modèles Numérique de Surface ainsi que sur les orthoimages. Nous montrons à la fin des résultats de classifications faites sur une régions de $502$ bâtiments hétéroclites dans la ville d'\'Elancourt.
    \end{abstract}
    \section{Introduction}
    \section{\'Etat de l'art}
    \section{Taxonomy d'erreurs}
    \section{Méthode proposée}
    \section{Conclusion et perspectives}
    \bibliography{references}
\end{document}
